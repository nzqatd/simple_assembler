\documentclass[a4paper,11pt,titlepage]{ltjsarticle}%ltjsarticleを使用

% 数式
\usepackage{amsmath,amsfonts}
\usepackage{bm}
\usepackage{luatexja}
% 画像
\usepackage{graphicx}
\usepackage{color}
% コード
\usepackage{listings,jvlisting}


\begin{document}

%ここから本文

\section{アセンブラについて}
本資料は、計算機科学実験及演習3で利用可能なSIMPLE向けアセンブラーに関するものです。命令に関する詳細は、別資料のSIMPLE 設計資料 (ver 4.0: 20200415)を参照してください。
\subsection{使い方}
テキストに命令を実行順で書き下します。引数については以下の表1の順で記載してください。
サンプルテキストとしてsample1.txtを用意してありますのでそちらもご確認ください。
インプットするファイルが準備できたら、
\begin{lstlisting}[label=fuga]
$python3 assembler.py input-file [output-file]
\end{lstlisting}
で、当プログラムを実行します。pythonのバージョンは3.7.6です。output-fileを省略した場合は標準出力に出力されます.
\begin{table}[htbp]
\centering
\caption{input-file内の引数の表記順}
\begin{tabular}{|c|c|c|c|}
\hline
\textbf{命令}  & \textbf{第一引数} & \textbf{第二引数} & \textbf{第三引数} \\ \hline
\textbf{ADD}  & R{[}Rd{]}     & R{[}Rs{]}     &               \\ \hline
\textbf{IADD} & R{[}Rd{]}     & R{[}Rs{]}     & d             \\ \hline
\textbf{SUB}  & R{[}Rd{]}     & R{[}Rs{]}     &               \\ \hline
\textbf{ISUB} & R{[}Rd{]}     & R{[}Rs{]}     & d             \\ \hline
\textbf{AND}  & R{[}Rd{]}     & R{[}Rs{]}     &               \\ \hline
\textbf{IAND} & R{[}Rd{]}     & R{[}Rs{]}     & d             \\ \hline
\textbf{OR}   & R{[}Rd{]}     & R{[}Rs{]}     &               \\ \hline
\textbf{IOR}  & R{[}Rd{]}     & R{[}Rs{]}     & d             \\ \hline
\textbf{XOR}  & R{[}Rd{]}     & R{[}Rs{]}     &               \\ \hline
\textbf{IXOR} & R{[}Rd{]}     & R{[}Rs{]}     & d             \\ \hline
\textbf{CMP}  & R{[}Rd{]}     & R{[}Rs{]}     &               \\ \hline
\textbf{MOV}  & R{[}Rd{]}     & R{[}Rs{]}     &               \\ \hline
\textbf{SLL}  & R{[}Rd{]}     & d             &               \\ \hline
\textbf{SLR}  & R{[}Rd{]}     & d             &               \\ \hline
\textbf{SRL}  & R{[}Rd{]}     & d             &               \\ \hline
\textbf{SRA}  & R{[}Rd{]}     & d             &               \\ \hline
\textbf{IN}   & R{[}Rd{]}     &               &               \\ \hline
\textbf{OUT}  & R{[}Rs{]}     &               &               \\ \hline
\textbf{HLT}  &            &               &               \\ \hline
\textbf{LD}   & R{[}Ra{]}     & d             & R{[}Rb{]}     \\ \hline
\textbf{ST}   & R{[}Ra{]}     & d             & R{[}Rb{]}     \\ \hline
\textbf{LI}   & R{[}Rb{]}     & d             &               \\ \hline
\textbf{B}    & d             &               &               \\ \hline
\textbf{BE}   & d             &               &               \\ \hline
\textbf{BLT}  & d             &               &               \\ \hline
\textbf{BLE}  & d             &               &               \\ \hline
\textbf{BNE}  & d             &               &               \\ \hline
\textbf{IBE}  & R{[}Rd{]}     & R{[}Rs{]}     & d             \\ \hline
\textbf{IBLT} & R{[}Rd{]}     & R{[}Rs{]}     & d             \\ \hline
\textbf{IBLE} & R{[}Rd{]}     & R{[}Rs{]}     & d             \\ \hline
\textbf{IBMT} & R{[}Rd{]}     & R{[}Rs{]}     & d             \\ \hline

\end{tabular}
\end{table}

\end{document}